% \iffalse
\let\negmedspace\undefined
\let\negthickspace\undefined
\documentclass[journal,12pt,twocolumn]{IEEEtran}
\usepackage{cite}
\usepackage{amsmath,amssymb,amsfonts,amsthm}
\usepackage{algorithmic}
\usepackage{graphicx}
\usepackage{textcomp}
\usepackage{xcolor}
\usepackage{txfonts}
\usepackage{listings}
\usepackage{enumitem}
\usepackage{mathtools}
\usepackage{gensymb}
\usepackage{comment}
\usepackage[breaklinks=true]{hyperref}
\usepackage{tkz-euclide} 
\usepackage{listings}
\usepackage{gvv}                                        
\def\inputGnumericTable{}                                 
\usepackage[latin1]{inputenc}                                
\usepackage{color}                                            
\usepackage{array}                                            
\usepackage{longtable}                                       
\usepackage{calc}                                             
\usepackage{multirow}                                         
\usepackage{hhline}                                           
\usepackage{ifthen}                                           
\usepackage{lscape}
\usepackage{amsmath}
\usepackage{caption}

\newtheorem{theorem}{Theorem}[section]
\newtheorem{problem}{Problem}
\newtheorem{proposition}{Proposition}[section]
\newtheorem{lemma}{Lemma}[section]
\newtheorem{corollary}[theorem]{Corollary}
\newtheorem{example}{Example}[section]
\newtheorem{definition}[problem]{Definition}
\newcommand{\BEQA}{\begin{eqnarray}}
\newcommand{\EEQA}{\end{eqnarray}}
\newcommand{\define}{\stackrel{\triangle}{=}}
\theoremstyle{remark}
\newtheorem{rem}{Remark}
\begin{document}

\bibliographystyle{IEEEtran}
\vspace{3cm}

\title{NCERT 10.5.3 Q16}
\author{EE23BTECH11038 - Rohith Madhani$^{*}$% <-this % stops a space
}
\maketitle
\newpage
\bigskip
\renewcommand{\thefigure}{\theenumi}
\renewcommand{\thetable}{\theenumi}

\textbf{Question :} A sum of Rs.700 is to be used to give seven cash prizes to students of a school for their overall academic performance. If each prize is 
Rs.20 less than its preceding prize, find the value of each of the prizes.

\solution 

\begin{table}[!h] 
\centering
\input{Tables/Table1}
\caption{Given parameters}
\label{table:Parameters list}
\end{table}

Taking Z transform of x(n),

\begin{align}
    X(z) &= \frac{x(0)}{1-z^{-1}} + \frac{20.z^{-1}}{(1-z^{-1})^2} ; |z|>1 \\
    \because y(n) &= x(n)*u(n) \\
    Y(z) &= X(z)U(Z) \\
    Y(z) &= \frac{x(0)}{(1-z^{-1})^2} + \frac{20.z^{-1}}{(1-z^{-1})^3}
\end{align}

Using contour integration for inverse Z transformation, 

\begin{align}
    y(6) &= \frac{1}{2\pi j}\oint_c Y(z) z^{5} dz \\
    &= \frac{1}{2\pi j}\int \frac{x(0).z^{7}}{(z-1)^{2}} dz + \frac{1}{2\pi j}\int \frac{20.z^{7}}{(z-1)^{3}} dz\\
    \because R&=\frac{1}{\brak {m-1}!}\lim\limits_{z\to a}\frac{d^{m-1}}{dz^{m-1}}\brak {{(z-a)}^{m}f\brak z}\\
    R_1 &=\frac{1}{1!}\lim\limits_{z\to 1}\frac{d}{dz}\brak {(z-1)^2.\frac{x(0).z^{7}}{(z-1)^{2}}}\\
    &= 7 \times x(0)\\
    R_2 &=\frac{1}{2!}\lim\limits_{z\to 1}\frac{d^2}{dz^2}\brak {(z-1)^3.\frac{20.z^{7}}{(z-1)^{3}}}\\
    &= 420\\
    \implies y(6) &= R_1 + R_2 \\
    700 &= 420 + 7 \times x(0) \\
    \therefore x(0) &= 40
\end{align}

The value of each of the prizes is 
\begin{align}
    x(0) &= 40 \\
    x(1) &= 40 + 20(1) = 60 \\
    x(2) &= 40 + 20(2) = 800 \\
    x(3) &= 40 + 20(3) = 100 \\
    x(4) &= 40 + 20(4) = 120 \\
    x(5) &= 40 + 20(5) = 140 \\
    x(6) &= 40 + 20(6) = 160 
\end{align}

\begin{figure}[!h] 
\centering
\includegraphics[width=\columnwidth]{figs/term.png}
\caption{x(n)=(40+20n)u(n)}
\label{fig:term_graph}
\end{figure}

\begin{figure}[!h] 
    \centering
    \includegraphics[width=\columnwidth]{figs/Sum.png}
    \caption{$y(n)=(50n+10n^2 )u(n)$}
    \label{fig:sum_graph}
    \end{figure}

\end{document}
